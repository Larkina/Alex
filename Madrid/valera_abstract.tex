\documentclass[12pt,a4paper]{article}
%\documentclass[17pt,russian]{extarticle}
\usepackage[utf8]{inputenc}
\usepackage[russian]{babel}
\usepackage[OT1]{fontenc}
\usepackage{amsmath}
\usepackage{amsfonts}
\usepackage{amssymb}
\usepackage{graphicx}
\usepackage{mathtools}
\usepackage[left=2cm, right=2cm, top=2cm, bottom=2cm]{geometry}

 \newcommand{\tit}[1]{\begin{center}{\bf{\Large #1}}\end{center}}
 \newcommand{\aut}[1]{\centerline{{\bf #1}}}
 \newcommand{\cityorg}[1]{\centerline{\it #1}}
 \newcommand{\email}[1]{\centerline{{\small e-mail: #1}}\vspace{\baselineskip}}
 \renewcommand{\normalsize}{\fontsize{16pt}{16}\selectfont}
 \renewcommand{\baselinestretch}{1.5}
\providecommand{\keywords}[1]{\textbf{\textit{Keywords:}} #1}

\begin{document}

We consider 3-D acoustic waves scattering model on permeable heterogeneous obstacle in bounded domain $\Omega$ of $\mathbb{R}^3$ corresponds to conjugation problem for 3-D Helmholtz 
equation with variable coefficients. Inverse problem which arises when constructing 
cloak devices for material bodies from observation by acoustic waves is formulated. 
This problem consists of finding one or two Helmholtz equation coefficients, describing 
variable medium properties for specific information about scattered acoustic field.
Using the optimization method, it reduces to the study of the inverse extremal problem,
which
consists of finding minimum of specific quality functional, depends on the control and
initial weak solution of boundary value problem. Solvability of inverse extreme problem
is proved.

Existence of Lagrange multiplier for every control problem is proved, optimality system 
including direct problem for main state, conjugate problem for the conjugate state
and variational inequality for finding desired controls is constructed.
Based on optimality system sufficient conditions for initial data are established.
This conditions provide uniqueness and stability of specific extreme problem solutions.
Furthermore, effective numerical algorithms for solving problem are developed.
\end{document} 