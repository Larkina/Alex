\documentclass[12pt,a4paper]{article}
%\documentclass[17pt,russian]{extarticle}
\usepackage[utf8]{inputenc}
\usepackage[russian]{babel}
\usepackage[OT1]{fontenc}
\usepackage{amsmath}
\usepackage{amsfonts}
\usepackage{amssymb}
\usepackage{graphicx}
\usepackage{mathtools}
\usepackage[left=2cm, right=2cm, top=2cm, bottom=2cm]{geometry}

 \newcommand{\tit}[1]{\begin{center}{\bf{\Large #1}}\end{center}}
 \newcommand{\aut}[1]{\centerline{{\bf #1}}}
 \newcommand{\cityorg}[1]{\centerline{\it #1}}
 \newcommand{\email}[1]{\centerline{{\small e-mail: #1}}\vspace{\baselineskip}}
 \renewcommand{\normalsize}{\fontsize{14pt}{14}\selectfont}
 \renewcommand{\baselinestretch}{1.5}
\providecommand{\keywords}[1]{\textbf{\textit{Keywords:}} #1}

\begin{document}§

\begin{center}
Сontrol approach in cloaking problems for Helmholtz and Maxwell
equations
\end{center}

In this paper control problems for Helmholtz and Maxwell equations associated with
acoustic and electromagnetic cloaking are considered.

The paper consists of two parts. In the first part the control problems are considered for
the 2-D Helmholtz equation describing scattering TH(or TE) polarized electromagnetic waves in
unbounded homogeneous medium containing a permeable dielectric obstacle with partially coated 
(for masking) boundary. These problems arise when creating methods of cloaking material
bodies. The role of control in control problem under study is played by boundary impedance
or boundary conductivity on the coated part of the boundary. Solvability of control problems is proved,
the optimality system which describes the necessary conditions of extremum is derived.
Based on its analysis the uniqueness and stability of optimal solutions are
established.

In the second part these results are generalized to the case of 3-D Maxwell equations considered
under impedance boundary conditions on the coated part of the boundary.

The work was supported by the Russian Foundation for Basic Research(project no. 13-01-00313-a), the Far East Branch
of the Russian Academy of Sciences (project no. 12-I-P17-03).

\end{document} 