\documentclass[12pt,a4paper]{article}
\usepackage[utf8]{inputenc}
\usepackage[russian]{babel}
\usepackage[OT1]{fontenc}
\usepackage{amsmath}
\usepackage{amsfonts}
\usepackage{amssymb}

 \newcommand{\tit}[1]{\begin{center}{\bf{\Large #1}}\end{center}}
 \newcommand{\aut}[1]{\centerline{{\bf #1}}}
 \newcommand{\cityorg}[1]{\centerline{\it #1}}
  \newcommand{\email}[1]{\centerline{{\small e-mail: #1}}\vspace{\baselineskip}}
\begin{document}

\sloppy

 \tit{Title goes here}
 \tit{Mantle cloak: Invisibility induced by a surface}
 \aut{Andrea Alu}
 \cityorg{Department of Electrical and Computer Engineering, 
 University of Texas at Austin,}
 \cityorg{1 University Station C0803, Austin,  Texas 78712, USA}
 \email{alu@mail.utexas.edu}

\begin{abstract}
Недавно для различных задач маскировки были применины экзотические взаимодействия волн
метаматериалов, но реализация метаматериалов в практической маскировке еще далека от
идеала. Текущие методы изготовления по своей природе основаны на объемных свойствах
метаматериалов, которые требуют хоть сколько-нибудь заметную электрическую толщину.
Я представляю здесь идею поверхности маскировки, показывая, что узорчатые 
метаматериалы могут давать те же эффекты маскировки в более простой и более тонкой 
геометрии. Токи, порожденные на неактивной поверхности, служат для резкого подавления 
видимости данного объекта. 
\end{abstract}

\end{document}