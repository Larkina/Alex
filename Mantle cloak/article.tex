\documentclass[12pt,a4paper]{article}
\usepackage[utf8]{inputenc}
\usepackage[russian]{babel}
\usepackage[OT1]{fontenc}
\usepackage{amsmath}
\usepackage{amsfonts}
\usepackage{amssymb}
\usepackage{graphicx}
\usepackage{mathtools}
\usepackage[left=2cm, right=2cm, top=2cm, bottom=2cm]{geometry}

 \newcommand{\tit}[1]{\begin{center}{\bf{\Large #1}}\end{center}}
 \newcommand{\aut}[1]{\centerline{{\bf #1}}}
 \newcommand{\cityorg}[1]{\centerline{\it #1}}
  \newcommand{\email}[1]{\centerline{{\small e-mail: #1}}\vspace{\baselineskip}}
\begin{document}

\sloppy

 \tit{Title goes here}
 \tit{Mantle cloak: Invisibility induced by a surface}
 \aut{Andrea Alu}
 \cityorg{Department of Electrical and Computer Engineering, 
 University of Texas at Austin,}
 \cityorg{1 University Station C0803, Austin,  Texas 78712, USA}
 \email{alu@mail.utexas.edu}

\begin{abstract}
Недавно для различных задач маскировки были применины экзотические взаимодействия волн
метаматериалов, но реализация метаматериалов в практической маскировке еще далека от
идеала. Текущие методы изготовления по своей природе основаны на объемных свойствах
метаматериалов, которые требуют хоть сколько-нибудь заметную электрическую толщину.
Я представляю здесь идею поверхности маскировки, показывая, что узорчатые 
метаматериалы могут давать те же эффекты маскировки в более простой и более тонкой 
геометрии. Токи, порожденные на неактивной поверхности, служат для резкого подавления 
видимости данного объекта. 
\end{abstract}

\setcounter{secnumdepth}{5}
\section{Введение}
Последние исследования в технологии метаматериалов показали, что невидимость,
прозрачность и маскировка могут быть получены разными способами, основанными на
сложном взаимодействии волн искуственных материалов и метаматериалов (смотори 
\cite{1,2}). Основанная на преобразованиях маскировка \cite{3}-\cite{6} является
самой популярной техникой, недавно была предпринята попытка расширить 
эксперементальную реализацию до видимых частот \cite{7}. Принциы работы такой 
маскировки заключен в электромагнитных свойствах объемных метаматериалов с заданным
спецефичным анизотропным и неоднородным профилем, который может направлять 
электромагнитные волны вокруг заданной области пространства, изолируя и делая
невидимым любой объект, помещенный в такую область. Другим жизнеспособным методом
маскировки является плазмонная \cite{8,9}, основанная на аннулировании рассеяния 
особенностей низко-проницаемых метаматериалов, которые могут быть поляризованы 
необычными способами, а так же аномально локализованные резонансные мезанизмы 
\cite{10}, основанные на квазистатических резонансных свойствах метаматериалов, 
которые могут эффективно маскировать заданную область. Все эти техники, так же как
и многие другие, включающие плащ из метаметериалов, основаны на специфичных объемных 
свойствах слоев метаматериалов.
В общем, эти искуственные метаматриалы основаны на коллективном электромагнитном
ответе составляющих их включений, которые взаимодействуют с падающими 
электромагнитными волнами как объем, получая эффект, кардинально отличающийся от
эффекта, получаемого от индивидуальных материалов, из которых они составлены.
С одной стороны, это может давать большую степень свободы для получения анольмальных
эффектов, таких как маскировка, с другой стороны плащи из метаматериалов изначально 
требуют определенной тонкости, из-за конечного размера составяющих включений. 
В случае с основанными на преобразованииплащами плащами, в частности, вовлеченный 
неоднородный профиль требует плащ, который имеет толщину, сравнимую по размеру с 
маскируемой областью. Более того, обычно требуется некоторое пространтсво между
плащом из метаматериалов и маскируемым объектом, чтобы гарантировать, что 
зернистость материала не порождает нежелаемых сцепок с объектом, которые могут 
повлиять на его электромагннитный свойства в целом \cite{11}. Более тонкий плащ
это не только непрактично и нежелательно, но также означает уменьшение пропускной
способности и увеличение чувствительности \cite{12}. Даже техника плазмонной 
маскировки, которая требует относительно тонкого плаща, по сравнению с основанными
на преобразовании метаматериалами, может требовать на практике конечной толщины
для надлежащей работы \cite{11,13}.

В другой области, понятие узорной тонкой металлической поверхности хорошо известно
в различных инжинерных приложениях, с соответствующими книгами и обзорами по этой
теме (смотри \cite{14}). При условии, что периодичесеский рисунок на металлической
поверхности намного меньше, чем длина волны, ее электромагнтиное поведение может
быть эффективно описано через усредненный импеданс поверхности $Z_s = R_s - iX_s$,
который связывает среднее тангенциальное электрическое поле на поверхности c 
средней плотностью индуцированного электрического тока как $\textbf{E}_{tan}=
Z_s\textbf{J}$. Импеданс $Z_s$ обычно предполагает широкий диапазон значений, как
функция пространства и частоты, из которой происходит название 
``Частотно-изберательная поверхность''(frequency selective surface (FSS)). 
В лучшем случае, легко показать, что $Z_s$ чисто мнимое, $R_s$
относится только к поглащению. В более общем случае, однако, значение $Z_s$ может зависеть 
от ориентации
тангенциального электрического поля, предполагающую анизотропную и тензорную структуру 
$\underline{\textbf{Z}}_s$. Скалярная запись может быть приемлимой для особых поляризаций падающей волны.

Далее, я покажу что единичной узурчатой FSS может быть достаточно, чтобы произвести эффект маскировки,
аналогичный эффекту с плащом из метаматериалов, даже в идеальном пределе с поверхностью нулевой толщины.
Это может породить тонкие плащи в полученных технологих с длинными историями применений,
обещает более легкую реализацию, возможное прилегание к форме объекта, низкий профиль и относительно
больший диапазон работы. Схожие идеи могут быть расширены на металлические поверхнгости тетрагерцевых
и оптических частот, открытие перспектив для реализации тонких плащей с повышенной производительностью.
Мантевая маскировка (\textit{mantle cloak}), которая предложена здесь может приблизить нас к практической
реализации маскировки, так как она не будет опираться на свойства, определяющих материал, а только на
поперечное сопротивление узорной металлической поверхности. Следует упомянуть, что мягкий и трудный
FSS уже применялисть в прошлом в устройствах снижения рассеяния \cite{15}, но использовали идеи,
решительным образом отличающиеся от примененных здесь. Здесь же, интерес заключается в реализации
механизма маскировки, который не зависит от угла падения и, возможно, от поляризации волны, что будет
описано далее.

\begin{figure}[t]
  \centering
  \includegraphics[height=0.15\paperheight]{1.png}
  \caption{Примеры узорных геометрических форм, которые могут реализовать мантевую маскировку.}
  \label{fig:1}
\end{figure}

\section{Теоретические формулировки}
Рассмотрим примеры геометрических форм в рисунке (\ref{fig:1}), т.е. диалектрические сферы 
радиусом $a$,
покрытые узорчатой металлической сферической поверхностью незначительной толщины радиуса $
a_c > a$.
Они представляют собой типичные шаблоны, которые могут быть реализованы на металлической поверхности
для получения квазиоднородного поверхностного реактивного сопротивления заданного значения. Доказано, что
шаблоны являются субволновыми, они могут создать квазиоднородное поверхнгостное реактивное сопротивление.
заданного значения.

\begin{figure}[t]
  \centering
  \includegraphics[height=0.3\paperheight, width=0.4\paperwidth]{2.png}
  \caption{Изменение в общем рассеяния поперечного сечения непроводящей сферы с $\varepsilon
  =10$ и $2a=\lambda_0/5$ с: (a) поверхностным реактивным сопротивлением mantle cloak; (b)
  нормализованной частотой для операций на плаще с $a_c=1.1a$ и $X_s=175\Omega$}
  \label{fig:2}
\end{figure}

Один предельный случай возникает, когда проводящая поверхность не имеет отверстий, которые мы могли
обеспечить эффективное нулевое тангенциальное електрическое поле, дающее $X_s = 0$. В другом крайнем
случае, когда метал отсутствует вовсе, поверхностное 
реактивное сопротивление $X_s \to \infty$.
Как показано в \cite{14} и находящихся в ней ссылках, надлежащий выбор шаблонов на мателлической
поверхности позволяет достигнуть желаемого положительного или отрицательного значения $X_s$ на 
интересуемой частоте. После гомогенизации задача рассеяния заданного произвольного возбуждения может
быть решена аналитически, введением желаемого скачка касательного магнитного поля на поверхности плаща
в $r=a_c$, пропорциольного среднему току, индуцированному на поверхности. Это означает, что граничное
условие 
\begin{equation}
\left.\textbf{H}_{tan}\right|_{r=a_c^+} - \left.\textbf{H}_{tan}\right|_{r=a_c^-} = 
\hat{r} \times \left.\textbf{E}_{tan}\right|_{r=a_c} / Z_s
\end{equation} 	
выполняется в изотропном случае. Решение Ми этой задачи означает, что н-тая 
поперечно-магнитная(Transverse-Magnetic) сферическая гармоника может быть подавлена при условии,
что следущий определитель аннулируется(?):
\begin{equation}\label{eq2}
\begin{vmatrix}
j_n(ka) & j_n(k_0a) & y_n(k_0a) & 0\\
[kaj_n(ka)]'/\varepsilon & [k_0aj_n(k_0a)]' & [k_0ay_n(k_0a)]' & 0\\
0 & j_n(k_0a_c)+\frac{[k_0a_cj_n(k_0a_c)]'}{iw\varepsilon_0a_cZ_s} &
y_n(k_0a_c) + \frac{[k_0a_cy_n(k_0a_c)]'}{iw\varepsilon_0a_cZ_s} & j_n(k_0a_c)\\
0 & [k_0a_cj_n(k_0a_c)]' & [k_0a_cy_n(k_0a_c)]' & [k_0a_cj_n(k_0a_c)]'
\end{vmatrix},
\end{equation}
где $j_n(\cdot)$ и $y_n(\cdot)$ сферические функции Бесселя, $k$ и $k_0$ волновые числа в объекте и
своободном пространтсве соотвественно, $\varepsilon$ --- диалектрическая проницаемость объекта,
$\varepsilon_0$ --- диалектрическая проницаемость свободного пространства.
Для поперечно-электрической(transverse-electric(TE)) гармоники можно легко 
получить сопряженное к уравнению ($\ref{eq2}$). При условии, что детерминант в уравнении 
(\ref{eq2}) может быть приближен к нулю для старших порядков рассеяния, видимость заданного
объекта будет резко уменьшена, в независимости от поляризации, вида возбуждений и позиции
наблюдателя, при этом достигается хоть и не идеальный, из-за остаточных членов рассеяния,
эффект маскировки. Может быть так же предусмотрено расширение до анизотропных поверхностей
и тензора $\underline{\textbf{Z}}_s$, производящее в целом поперечное соединение двух 
поляризаций. Следует заметить, однако, что изотропная формулировка в (\ref{eq2}) применима
даже для анизотропных поверхностей с спецефическими поляризациями входящей волны.

\begin{figure}[t]
  \centering
  \includegraphics[height=0.16\paperheight, width=0.4\paperwidth]{3.png}
  \caption{Распределение: (а)амлитуды электрического поля в H плоскости;(b)фазы магнитного 
  поля на E плоскости для трех случаев: замаскированной сферы(слева), той же сферы, но не 
  замаскированной(по центру), увеличенной незамаскированной непроводящей сферы с $a=a_c$(
  справа). Более яркие цвета соответствуют большим значениям полей. Плоскости изображены в 
  одинаковом масштабе для справедливого сравнения.}
  \label{fig:3}
\end{figure}

Полезно проанализировать формулу (\ref{eq2}) в квазистатическом пределе(электрически 
маленькие объекты), 
для которых $(k_0a_c) \ll 1$. В этом случай основной вклад в рассеяние дает
$n=1$ доминантная гармоника и приближенные условия для маскировки для маскировки в 
случае двух поляризаций можно записать в явном виде как
\begin{equation*}
TM:X_s = 
\frac{2[2+\varepsilon-\gamma^3(\varepsilon-1)]}{3\gamma^3\omega a\varepsilon_0(\varepsilon-1)
}
\end{equation*}
\begin{equation}\label{eq3}
TE:X_s = \frac{\omega a\mu_0[2+\mu+2\gamma^3(\mu-1)]}{6\gamma^3(\mu-1)}
\end{equation}
где $\mu$ означает проницаемость, а $\gamma=a/a_c$.

Уравнение ($\ref{eq3}$) показывает, что в квазистатическом пределе вклад TE и TM гармоник в 
рассеяние разделяется, как и ожидалось, и это гарантирует, что доминирующие мультиполярные
члены, как эдектрические так и магнитные, оба могут быть подавлены, при правильном выборе
реактивного сопротивления FSS. Несмотря на то, что нулевое рассеяние образуется при помощи 
тонкой поверхности, конформной объекту, оно может быть достигнуто в этом пределе без потерь 
реактивного сопротивления, и наличие реалистичных потерь в металле не повлияет на эффект 
маскировки. Когда размер объекта увеличивается, динамические формулы как в (\ref{eq2}) могут
быть использованы для правильного построения плаща. Эта формулировка может быть расширена на
случай проводящих объектов и различных несимметричных геометрий и анизотропий без изменения
результата.

\begin{figure}[t]
  \centering
  \includegraphics[height=0.15\paperheight, width=0.4\paperwidth]{4.png}
  \caption{Аналогично рисунку (\ref{fig:2}), но для проводящей сферы того же размера}
  \label{fig:4}
\end{figure}

\section{Численные результаты}
Рассмотрим, как пример, непроводящую сферу с диалектрической проницаемостью $\varepsilon=10$ 
и диаметром $2a=\lambda_0/5$, где $\lambda_0$ --- длина волны в свободном пространстве. 
Рисунок (\ref{fig:2}а) показывает изменение общего рассеяния в поперечном сечении в 
зависимости от реактивного сопротивления $X_s$ плаща по сравнению с <<голой>> сферой
(тонкие пунктирные линии). Очевидно, что при достаточно больших значениях реактивного 
сопротивления узорная поверхность не оказывает никакого влияния на рассеяние (предел при
отсутствии поверхности задается $X_s \to \infty$), но для конкретных индуктивных значений,
качетсвенно согласующихся с уравнением (\ref{eq3}), хотя и с некоторым отклонением, из-за
большого электрического размера объекта, достигнуто соответствующее снижение рассеяния.
Это можно получить для различных значений $a_c$, даже в пределе с поверхностью конформной
объекту ($\gamma=1$). Замечено, что для других значений реактивного сопротивления может
быть получен сильный резонанс для различных порядков рассеяния, которые, хотя здесь и не
релевантны, могут быть полностью охарактеризованы с помощью этого анализа. Для случая
$a_c=1.1a$ может быть получено более 97\% уменьшения рассеяния при реактивном сопротивлении
$175\Omega$.

Как показано на рисунке (\ref{fig:2}a), поверхностная маскировка может давать хорошие 
показатели для относительно широкого диапазона сопротивлений, основываясь на механизме
аннулирования рассеяния, в некотором смысле аналогично технике плазмонной маскировки 
\cite{8}, хотя основана не на отрицательной поляризации покрытия метаматериалом, а на
конкретных шаблонах тока, индицированного на FSS, и полей, возникающих между поверхностью и
объектом. Это объясняет надежность и пропускную способность этого механизма маскировки,
несмотря на крайне низкий профиль плаща \cite{13}. Рисунок (\ref{fig:2}b) показывает
частоту ответа плаща на рисунке (\ref{fig:2}a) с $a_c=1.1a$ и $X_s = 175\Omega$, здесь 
предполагается, что поверхностное реактивное сопротивление постоянно с частотой. Это 
предположение может считаться действительным в диапазоне частот выокруг построенной
частоты $f_0$. Отметим, что значительное снижение рассеяния может быть достигнуто, по 
сравнению с незамаскированным случаем (штрихованная красная линия) или незамаскированной
частицей с $a=a_c$ (пунктирная голубая линия). Ясно, что ответ FSS не может считаться 
полностью недисперсионным в широком диапазоне частот, но отметим, что благодаря отсутствию
объемных полей ограничения на дисперсию более слабые, чем в типичных моделях дисперсии для 
метаматериалов. Это отражает, что поверхностное реактивное сопротивление слабо диссипативно
с частотой и имеет предсказуемо большой диапазон работы, в соответствии с нашими результатами
для полных колебаний, которые представленны ниже.

\begin{figure}[t]
  \centering
  \includegraphics[height=0.15\paperheight, width=0.4\paperwidth]{5.png}
  \caption{Аналогично рисунку (\ref{fig:3}), но для проводящей сферы такого же размера}
  \label{fig:5}
\end{figure}

Рисунок (\ref{fig:3}) показывает амлитуду распределения электрического поля в плоскости Н и
фазу разпределения магнитного поля в E плоскости для трех случаев: замаскированная сфера(
левая колонка); <<голая>> сфера(по центру); и <<голая>> сфера с $a=a_c$ для геометрии
рисунка (\ref{fig:2}b) на частоте $f_0$. Мы предпологали, что для каждой плоскости снизу
происходит возмущение падающими плоскими волнами. Различные плоскости приведены в одной и той
же цветовой гамме, для честного сравнения. Видно, что manlte cloak действитльно производит
сильное сокращение рассеяния, благодаря правильному выбору реактивного сопротивления 
поверхностного имеданса, восстанавливает почти равномерную амплитуду плоские фазовые фронты
вокруг всего плаща в обеих плоскостях поляризации, в соответствии с 97\% уменьшением
рассеяния, предсказанным на рисунке (\ref{fig:2}). Интересно, что видно, как поле может
проникать в поверхность и распостраняться внутри области между сферой и плащом, а так же в 
самой сфере, обеспечивая большую надежность и пропускную способность \cite{12}, по сравнению
с методами маскировки, которые направляют волну вокруг объекта. Это может также обеспечить
интригующую возможонсть улавливать и извлекать сигнал внутри поверхности без значительного
возмущения окружающих электромагнитных полей \cite{16}. Таким образом, mantle cloak может
предоставлять реальный способ для неинвазивного считывания данных и зондирования с улучшенной
пропускной способностью \cite{17}. Распределение поля на рисунке (\ref{fig:3}) также 
подчеркивает разницу между физическими механизмами, лежащими в основе этой маскировки и
техники плазмонной маскировки \cite{8}. Здесь увеличенное магнитное поле индуцируется 
около поверхности для поддержания шаблонов тока, тогда как электрическое поле максимально в
центре объекта, что разительно отличается от техники плазмонной маскировки, в которой 
магнитное поле непрерывно на поверхности плаща, в то время как электрическое поле максимально
около плаща, а не внутри диалектрического объекта. Это может давать дополнительные 
преимущества с точки зрения применения в зондировании такой техники маскировки.

Рисунки (\ref{fig:4}) и (\ref{fig:5}) демонстрируют схожие рисункам 
(\ref{fig:2}) и (\ref{fig:3})
моделирования, но для случая проводящей непроницаемой сферы того же размера. Очевидно, что
в этом случае волна не может проникнуть в объект, и поэтому требует ненулевое расстояние
между объектом и плащом. Это явствует из того, что в случае $a=a_c$ поверхностное реактивное
сопротивление будет замкнуто металлической поверхностью объекта. Тем не менее, даже узкий
промежуток между маскировочной поверхностью и объектом межет обеспечить надлежащую 
маскировку, как показано на рисунке (\ref{fig:4}). В качестве отступления, эта геометрия 
особенна интересна, так как может быть использована для реализации FSS маскировки
независимо от маскируемого объекта(путем помещения объекта внутри проводящей сферической 
полости). Как видно на рисунке (\ref{fig:4}), эта цена, которую следует заплатить, снижает
пропускную способность и несколько ухудшает производительность маскировки. Графики полей
на рисунке (\ref{fig:5}) отвечают случаю $a_c=1.1a$ с оптимальным реактивным сопротивлением,
как получено в уравнении (\ref{eq2}), $X_s=92\Omega$. В целом, снижение рассеяния в этом
случае составляет 86\%, немного меньше, чем в диэлектрическом случае, из-за сильного
вклада в рассеяние мод более выского порядка. Тем не менее, несмотря на непроницаемость 
объекта простая и тонкая реактивная поверхность может обеспечить резкое снижение видимости
во всех направлениях и обеих плоскостях поляризации.

Наконец, на графике (\ref{fig:6}) мы показываем моделирование полного колебания волны 
реалистичной квазиизотропной FSS, реализованной, чтобы скрыть объект (\ref{fig:2})(геометрия
поверхности показана на вставке в рисунке (\ref{fig:6})). Плащ образован шестью 
ортогональными
проводящими взаимосвязанными полосами, которые могут обеспечить квазиизотропный индуквтиный
отклик вокруг интересуюущей нас частоты $f_0$, для которой объект имеет размер около
$\lambda_0/3$. Эти полоски могут обеспечить требуемую усредненную изотропную проводимость
в трех плоскостях поляризации. Обратим внимание на относительно широкую пропускную 
способность mantle cloak, соответствующую предыдущим соображениям о возможности реализации
слабо рассеивающего поверхностного реактивного сопртивления. Более того, эта геометрия
квазиизотропна, так как его ответ слабо зависит от поляризации и угла падения, как видно
из ее конструкции. Простота дизайна этого плаща делает его особенно привлекательным для
объектов среднего размера. Расширение на большие электрические объекты потребует подавления
нескольких порядков рассеяния, что может потребовать нескольких слоев FSS или более сложных
сценариев. Мы убедились, что для диэлектрических объектов, вплоть до длин волн несколько
больших, значительное снижение рассеяния может быть достигнуто с одним слоем FSS, описанным
здесь.

\begin{figure}[t]
  \centering
  \includegraphics[height=0.2\paperheight, width=0.4\paperwidth]{6.png}
  \caption{Результаты моделирования для реалистичной квазиизотропной индуктивной FSS 
  маскировки, покрывающий непроводящую сферу с рисунка (\ref{fig:2})}
  \label{fig:6}
\end{figure}

\section{Заключение}
Я представил здесь идею маскировки с помощью тонких шаблонных порвехностей с низким профилем,
показывая, что токи и поля, индуцированные на FSS могут дать маскировку для диэлектрических
и проводящих объектов при помощи простой и практичной геометрии для относительно широкого
диапазона частот. Теоретический анализ подтвержден численными результатами и конструированием
реалистичной изотропной FSS-плаща для непроводящего шара. Схожие концепции могут быть 
транслированы на тетрагерцевые и оптические частоты, используя шаблонные плазмонные 
поверхности. В этом случае, предположительно лучшие результаты будут получены для 
поверхностей, более тонких, чем глубина верхнего слоя металла, для которых будет применен
анализ, схожий с проделанным в этой работе. Даже для более тонких поверхностей могут 
быть предусмотрены реалистичные эффекты маскировки на оптических частотах, если определить
поврехностный импеданс как функцию, зависящую от проводящих и токов смещения, индуцированных
на этой поверхности \cite{18}. Это может стать многообещающим для реализации оптической 
маскировки, для таких объектов как наночастицы и острие микроскопа \cite{16,17}. 
Предполагается так же, что эти идеи могут быть распостранены на каскадные реактивные 
поверхности, которые могут обеспечить работу в многих частотах сразу и/или заменить несколько
разработок для случая больших объектов. Это будет соответствовать \cite{9}, но теперь с 
использованием тонких шаблонных поверхностей с надлежащим ответом. Анализу этого будет
посвящена следущая работа автора.


\begin{thebibliography}{99}
\bibitem{1}A. Alu and N. Engheta, J. Opt. A 10, 093002 (2008).
\bibitem{2}P. Alitalo and S. Tretyakov, Mater. Today 12, 22 2009.
\bibitem{3}J. B. Pendry, D. Schurig, and D. R. Smith, Science 312, 1780 2006.
\bibitem{4}D. Schurig, J. J. Mock, B. J. Justice, S. A. Cummer, J. B. Pendry, A. F. Starr, 
and D. R. Smith, Science 314, 977 2006.
\bibitem{5}W. Cai, U. K. Chettiar, A. V. Kildishev, and V. M. Shalaev, Nat.
Photonics 1, 224 2007.
\bibitem{6}U. Leonhardt, Science 312, 1777 2006.
\bibitem{7}J. Valentine, J. Li, T. Zentgraf, G. Bartal, and X. Zhang, Nature
Mater. 8, 568 2009.
\bibitem{8}A. Alù and N. Engheta, Phys. Rev. E 72, 016623 2005.
\bibitem{9}A. Alù and N. Engheta, Phys. Rev. Lett. 100, 113901 2008.
\bibitem{10}G.W. Milton and N.-A. P. Nicorovici, Proc. R. Soc. London, Ser.A 462, 3027 
2006.
\bibitem{11}M. G. Silveirinha, A. Alù, and N. Engheta, Phys. Rev. E 75,
036603 2007.
\bibitem{12}A. Alù and N. Engheta, Phys. Rev. E 78, 045602(R) 2008.
\bibitem{13}A. Alù and N. Engheta, Opt. Express 15, 3318 2007.
\bibitem{14}B. A. Munk, Frequency Selective Surfaces: Theory and Design John Wiley \& Sons,
 New York, 2000.
\bibitem{15}P.-S. Kildal, 19th International Conference on Applied Electromagnetics
and Communications, 2007, ICECom 2007, 24–26 Sept. 2007, pp. 1–4.
\bibitem{16}A. Alù and N. Engheta, Phys. Rev. Lett. 102, 233901 2009.
\bibitem{17}F. J. García de Abajo, Phys. 2, 47 2009.
\bibitem{18}D. Van Labeke, D. Gérard, B. Guizal, F. I. Baida, and L. Li, Opt.
Express 14, 11945 2006.
\end{thebibliography}
\end{document} 