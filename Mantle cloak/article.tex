\documentclass[12pt,a4paper]{article}
\usepackage[utf8]{inputenc}
\usepackage[russian]{babel}
\usepackage[OT1]{fontenc}
\usepackage{amsmath}
\usepackage{amsfonts}
\usepackage{amssymb}

 \newcommand{\tit}[1]{\begin{center}{\bf{\Large #1}}\end{center}}
 \newcommand{\aut}[1]{\centerline{{\bf #1}}}
 \newcommand{\cityorg}[1]{\centerline{\it #1}}
  \newcommand{\email}[1]{\centerline{{\small e-mail: #1}}\vspace{\baselineskip}}
\begin{document}

\sloppy

 \tit{Title goes here}
 \tit{Mantle cloak: Invisibility induced by a surface}
 \aut{Andrea Alu}
 \cityorg{Department of Electrical and Computer Engineering, 
 University of Texas at Austin,}
 \cityorg{1 University Station C0803, Austin,  Texas 78712, USA}
 \email{alu@mail.utexas.edu}

\begin{abstract}
Недавно для различных задач маскировки были применины экзотические взаимодействия волн
метаматериалов, но реализация метаматериалов в практической маскировке еще далека от
идеала. Текущие методы изготовления по своей природе основаны на объемных свойствах
метаматериалов, которые требуют хоть сколько-нибудь заметную электрическую толщину.
Я представляю здесь идею поверхности маскировки, показывая, что узорчатые 
метаматериалы могут давать те же эффекты маскировки в более простой и более тонкой 
геометрии. Токи, порожденные на неактивной поверхности, служат для резкого подавления 
видимости данного объекта. 
\end{abstract}

\setcounter{secnumdepth}{5}
\section{Введение}
Последние исследования в технологии метаматериалов показали, что невидимость,
прозрачность и маскировка могут быть получены разными способами, основанными на
сложном взаимодействии волн искуственных материалов и метаматериалов (смотори 
\cite{1,2}). Основанная на преобразованиях маскировка \cite{3}-\cite{6} является
самой популярной техникой, недавно была предпринята попытка расширить 
эксперементальную реализацию до видимых частот \cite{7}. Принциы работы такой 
маскировки заключен в электромагнитных свойствах объемных метаматериалов с заданным
спецефичным анизотропным и неоднородным профилем, который может направлять 
электромагнитные волны вокруг заданной области пространства, изолируя и делая
невидимым любой объект, помещенный в такую область. Другим жизнеспособным методом
маскировки является плазмонная \cite{8,9}, основанная на аннулировании рассеяния 
особенностей низко-проницаемых метаматериалов, которые могут быть поляризованы 
необычными способами, а так же аномально локализованные резонансные мезанизмы 
\cite{10}, основанные на квазистатических резонансных свойствах метаматериалов, 
которые могут эффективно маскировать заданную область. Все эти техники, так же как
и многие другие, включающие плащ из метаметериалов, основаны на специфичных объемных 
свойствах слоев метаматериалов.
В общем, эти искуственные метаматриалы основаны на коллективном электромагнитном
ответе составляющих их включений, которые взаимодействуют с падающими 
электромагнитными волнами как объем, получая эффект, кардинально отличающийся от
эффекта, получаемого от индивидуальных материалов, из которых они составлены.
С одной стороны, это может давать большую степень свободы для получения анольмальных
эффектов, таких как маскировка, с другой стороны плащи из метаматериалов изначально 
требуют определенной тонкости, из-за конечного размера составяющих включений. 
В случае с основанными на преобразованииплащами плащами, в частности, вовлеченный 
неоднородный профиль требует плащ, который имеет толщину, сравнимую по размеру с 
маскируемой областью. Более того, обычно требуется некоторое пространтсво между
плащом из метаматериалов и маскируемым объектом, чтобы гарантировать, что 
зернистость материала не порождает нежелаемых сцепок с объектом, которые могут 
повлиять на его электромагннитный свойства в целом \cite{11}. Более тонкий плащ
это не только непрактично и нежелательно, но также означает уменьшение пропускной
способности и увеличение чувствительности \cite{12}. Даже техника плазмонной 
маскировки, которая требует относительно тонкого плаща, по сравнению с основанными
на преобразовании метаматериалами, может требовать на практике конечной толщины
для 	надлежащей работы \cite{11,13}.

В другой области, понятие узорной тонкой металлической поверхности хорошо известно
в различных инжинерных приложениях, с соответствующими книгами и обзорами по этой
теме (смотри \cite{14}). При условии, что периодичесеский рисунок на металлической
поверхности намного меньше, чем длина волны, ее электромагнтиное поведение может
быть эффективно описано через усредненный импеданс поверхности $Z_s = R_s - iX_s$,
который связывает среднее тангенциальное электрическое поле на поверхности c 
средней плотностью индуцированного электрического тока как $\textbf{E}_{tan}=
Z_s\textbf{J}$. Импеданс $Z_s$ обычно предполагает широкий диапазон значений, как
функция пространства и частоты, из которой происходит название ``Частотно-
изберательная поверхность``.

\end{document}