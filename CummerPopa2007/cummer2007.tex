\documentclass[a4paper, 12pt]{article}
\usepackage{cmap}
\usepackage[utf8]{inputenc}
\usepackage[english, russian]{babel}
\usepackage[left=2cm, right=2cm, top=2cm, bottom=2cm]{geometry}
\usepackage{amsfonts,amssymb}
\usepackage{amsmath}
\usepackage{amsthm}
\usepackage{titlesec}
\usepackage{graphicx}
\usepackage{mathtools}
\usepackage{hyperref}

\newcommand{\tit}[1]{\begin{center}{\bf{\Large #1}}\end{center}}
\newcommand{\aut}[1]{\centerline{{\bf #1}}}
\newcommand{\cityorg}[1]{\centerline{\it #1}}
\newcommand{\email}[1]{\centerline{{\small e-mail: #1}}\vspace{\baselineskip}}
\providecommand{\keywords}[1]{\textbf{\textit{Ключевые слова:}} #1}
\newcommand{\norm}[1]{\left\lVert#1\right\rVert}
\newcommand{\normb}[1]{\left\lVert\textbf{#1}\right\rVert}
\newcommand{\dbar}[1]{\bar{\bar{#1}}}

\begin{document}
/Alex/Perevod/Cloack/Cloack14/Cummer08.pdf

\sloppy
\tit{Scattering Theory Derivation of a 3D Acoustic Cloaking Shell}
\tit{Вывод теории рассеяния для трехмерной акустической маскирующей оболочки}
\aut{Steven Cummer}
\aut{Bogfan-Ioan Popa}

\begin{abstract}
При помощи акустической теории рассеяния мы выводим массовую плотность 
и модуль сжатия сферической оболочки, которая может устранять рассеяние 
произвольного объекта, находящегося во внутренности оболоки, по другому 
называющейся трехмерной  маскирующей оболочкой. 
Вычисления подтверждают, что поля давления и скорости 
плавно изгибаются и исключаются из центральной области, как для рассмотренной ранее
электромагнитной маскирующей оболочки. Оболочка требует анизотропной массовой 
плотности с главными осями в сферических координатах и зависящий от радиуса модуль
сжатия. Существование такой трехмерной оболочки означает, что такое нерассеивающие
решение может так же может существовать для других волновых систем, которые не 
изоморфны с электромагнитным случаем. 
\end{abstract}

Pendru el al. \cite{1} показали, что произвольное преобразование координат уравнений
Максвелла может быть интерпреировано в терминах электромагнитного материала в
исходной системе координат с преобразованными величинами диэлектрической и магнитной
проницаемости. Следовательно, изгибание и растягивание электромагнитных полей, 
определяемое преобразованием координат, 
может быть реализовано с помощью электромагнитных материалов, открывая неожиданные
и интересные решения, такие как электромагнитная маскировка \cite{1,2}.

Степень, с которой такой подход к маскировке может быть расширен для других 
классов волн, в общем не известна. Миллер \cite{3} описал активный подход для
общей маскировки волн, основанный на зондировании и нелокальной ретрансляции 
сигналов на поверхности объекта. Leonhardt \cite{4} описал как двумерный объект
может быть замаскирован в коротковолновом пределе способом, который так же может
быть применен к разным класса волн. Мильтон et al. \cite{5} показали, что 
подход преобразования координат не может быть расширен в общем случае до эластичной
среды. Cummer и Schuring \cite{6} показали, однако, что существует точная аналогия
медлу двумерным электромагнетизмом и акустикой для анизотропных материалов и, 
следовательно, что существует думерная маскировочная акустическая оболочка. 
Этот изоморфизм не расширяется на три измерения, однако, это означает, что если
трехмерная акустическая маскировочная оболочка существует, она отличается
от ее электромагнитной коллеги.

Chen et al. \cite{7} произвели анализ сферических рассеивающих гармоник 
трехмерной сферической маскирующей оболочки, описанной в \cite{1}  
и подтвердили, что такая оболочка делает любой объект в ее внутренности свободным 
от рассеяния во всех направлениях. Здесь мы используем анализ рассеяния как 
отправную точку для того чтобы вывести множество акустических материальных 
параметров для оболочки, которая обеспечивает трехмерному объекту во внутренности
оболочки свободу от акустического рассеяния.

Для невязкой жидкости с нулевым модулем скольжения динамика малых возмущений
описывается сохранением момента и отношением зависимости деформации от 
напряжения. Так как \cite{6} показали, что для двумерной маскировки требуется
анизотропная массовая плотность, мы полагаем эту анизотропию с самого начала.
С $\exp(-iwt)$ зависимостью от времени, эти уравнения движения выглядят как:

\begin{equation}\label{e1}
	\nabla p = iw\dbar{\rho}(\bar{r})\rho_0 \mathbf{v},
\end{equation}

\begin{equation}\label{e2}
	iwp=\lambda(\bar{r})\lambda_0 \nabla \cdot \mathbf{v},
\end{equation}
где $p$ ~--- скалярное давление, $\mathbf{v}$ ~--- вектор скорости жидкости,
$\lambda$ ~--- неодронодный модуль сжатия житкости, связанный с $\lambda_0$,
и $\dbar{\rho}$ есть неоднородный обобщенный тензор массовой плотности, связанный
с $\rho_0$. Эти уравнения являются версией для жидкости более общих
эластодинамических уравнений, рассмотренных в \cite{5}.
Хотя анизотропая массовая плотность не является свойством обычно присущим
натуральным материалам, она естественно возникает в анализе эластодинамики
сильно неоднородных композитных материалов \cite{8}. Важно отметить,
что анизотропная массовая плотность в \eqref{e1} является эффективной
массовой плотностью и не обязательно связана с физической массовой плотностью 
любого из индивидуальных элементов композиного материала. Milton et al. \cite{5}
описали простую концептуальную модель, осованную на массово загруженных
пружинах, встроенных в исходную матрицу, которая описывается анизотропной массовой
плотностью. Сила, примененная рядом с внутренней резонасной частотой такого
включения, произведет большие движения, в результате чего сообщая среде малую 
эффективную плотность масс. Если внутренние пружины изменяются, в зависимости от
направления, то эффективная массовая плотность так же будет зависеть от 
направления, тоесть будут анизотропными. Композитный материал, содержащий
анизотропные механически резонансные включения в исходной матрице жидкости
будут подчинены скалярному соотношению напряженности-деформаированости потому, что 
при применении сил матрица жидкости все еще будет иметь анизотропный динамический 
ответ, и следоватльно, анизотропную динамическую массовую плотность.

Мы рассматриваем эти поля в области, изображенной на Рис. 1, в которой
однородная жидкость изотропной плотность $\rho_0$ и модулем сжатия
$\lambda_0$ находится в $r>b$ и $r<a$, в то время как материал, находящийся в
 $a<r<b$ --- неоднородная анизотропная искомая оболочка. Без потери общности будем 
 считать, что одднородная плоская полна падает на оболочку
  по направлению $\theta=0$.

 Симметрия диктует, что если маскирующая оболочка существует, то ее параметры
 рассеяния должны не зависеть от направления падающей волны. Следовательно, 
 ее материальные свойства должны зависить только от радиуса, а главные оси
 тензора массовой плотности должны быть выровнены по направлению сферических
 координат. $\rho_r(r), \rho_\theta(r), \rho_\phi(r)$ и $\lambda(r)$ --- 
 материальные свойства, которые нужно определить. Более того, симметрия 
 предполагает, что $\rho_\theta(r)=\rho_\phi(r)$. Предполагается, что эти параметры
 нормализованы по отношению к плостности $\rho_0$ и модулю сжатия $\lambda_0$окружения.
\begin{thebibliography}{99}
\bibitem{1}J. B. Pendry,D. Schurig, and D. R. Smith, Science 312, 1780 (2006).
\bibitem{2}D. Schurig, J. B. Pendry, and D. R. Smith, Opt. Express 14,
9794 (2006).
\bibitem{4}U. Leonhardt, Science 312, 1777 (2006).

\end{thebibliography}
\end{document}
