\documentclass[12pt,a4paper]{article}
%\documentclass[17pt,russian]{extarticle}
\usepackage[utf8]{inputenc}
\usepackage[russian]{babel}
\usepackage[OT1]{fontenc}
\usepackage{amsmath}
\usepackage{amsfonts}
\usepackage{amssymb}
\usepackage{graphicx}
\usepackage{mathtools}
\usepackage[left=2cm, right=2cm, top=2cm, bottom=2cm]{geometry}

 \newcommand{\tit}[1]{\begin{center}{\bf{\Large #1}}\end{center}}
 \newcommand{\aut}[1]{\centerline{{\bf #1}}}
 \newcommand{\cityorg}[1]{\centerline{\it #1}}
 \newcommand{\email}[1]{\centerline{{\small e-mail: #1}}\vspace{\baselineskip}}
 \renewcommand{\normalsize}{\fontsize{16pt}{16}\selectfont}
 \renewcommand{\baselinestretch}{1.5}
\begin{document}

\sloppy

 \tit{Title goes here}
 \tit{Invisibility cloaking by coordinate transformation}
 \aut{Min Yan, Wei Yan and Min Qiu}
 \cityorg{Department of Microelectronics and Applied Physics,}
 \cityorg{Royal Institute of Technology (KTH), Electrum 229, 164 40 Kista, Sweden}
 \email{min@kth.se (M. Qiu)}

\section{Введение}
Появившиеся в последнее время искусственные электромагнитные (EM) материалы, называемыме 
\textit{метаматериалами} (Smith, Pendry and Wiltshire, 2004; Shalaev, 2006), открыли для нас
много способов взаимодействовать или управлять EM волнами. Были эксперементально продемонстрированы EM
явления, не существующие в природе. Отрицательный индекс преломления(Shelby, Smith and Schultz, 2001)
и отрицательный показатель пластины суперлинзы (Pendry, 2000) являются типичными примерами такого класса
спроектированных материалов. С помощью метаматериалов мы сповобны не только не только приспосабливать 
диалектрическую проницаемость и ее значения по желанию, на также точный контроль анизотропии 
метаматериалами и как зависят параметры от пространственного расположения. Хотя построение каждого 
отдельного блока метаматериала не является проблемой, помещение блоков в подходящем порядке, для 
достижения желаемого макроскопического оптического феномена остается загадкой. Говоря простым языком,
как с помощью деревьев различных цветов и размеров, имеющихся в нашем распоряжении, построить великолепный
лес? Недавно предложенный способ преобразования координат представляет собой идеальный рецепт для
решения такой задачи проектирования.

Данная статья посвящено разработке осбого типа EM устройства, мантии невидимки, полученной с помощью
техники преобразования координат. Структура нашей работы следущая: сначала, в разделе 2 обощается
теория преобразования координат для уравнения Максвелла. Принципы и построение скрывающих оболочек,
основанные на технике преобразования координат, представлены в разделе 3. В разделе 4 будет доказана
идеальная маскировка оболочек произвольной формы, используя доказаналитический анализ двухпериодной волны.
Будут внимательно рассмотрены физические параметры оболочек произвольной формы. После освещения
оболочек произвольной формы, мы будем детально изучать цилиндрические и сферические оболочки 
(секции 5 и 6). Специальное внимание будет уделено цилиндрической маскирующей оболочке, так как эта 
структура, возможно, является самой простой с точки зрения реализации. Некоторые практические вопросы,
касающиеся маскирующих оболочек, а также другие связанные исследования будут обсуждаться в разделе 7.
Наконец, в разделе 8 будут подведены итоги.

\section{Преобразование координат в электромагнетизме}
Теория трансформационной оптики уходит корнями в ковариционные свойства уравнений Максвелла. 
Лучшее математическое описание такого ковариционного свойства может быть дано с помощью диффиренциальной
геометрии (Post, 1962), аппарата, в общем используемого для разработки теории общей относительности
(Leonhardt and Philbin, 2006). Строгий вывод теории трансформационной оптики в четырехмерной пространстве
Минковского можно найти в Leonhardt and Philbin (2008). В этой работе мы напрямую используем наиболее
важные выводы в математически более доступной форме. Так как большинство приложений трансформационной 
оптики являются статическими или медленнодвижущимися по сравению с скоростью света, мы всегда можем
выбрать должным образом нашу рабочую область, и поэтому можно рассматривать только пространственное 
преобразование координат. В этой статье время не будет участвовать в преобразовании координат.

В плоском трехмерной евклидовом пространстве макроскопические уравнения Максвелла могут быть записаны как:

\begin{equation}\label{e1}
	\nabla \times \mathbf{E} = - \frac{\partial \mathbf{B}}{\partial t}, \qquad
	\nabla \times \mathbf{H} = \frac{\partial \mathbf{D}}{\partial t} + \mathbf{j}, \qquad
	\nabla \cdot \mathbf{D} = \rho, \;\;
	\nabla \cdot \mathbf{B} = 0.
\end{equation}
$\mathbf{E}$ и $\mathbf{H}$ --- электрические и магнитные поля, соответственно. $\mathbf{D}$ и 
$\mathbf{B}$ --- поток электрических и магнитных плотностей, соответственно. $\mathbf{j}$ плотность 
электрического тока, и $\rho$ плотность заряда. Уравнения Максвелла дополняются двумя соотношениями:

\begin{equation}\label{e2}
	\mathbf{D} = \epsilon_0 \bar{\bar{\varepsilon}} \cdot \mathbf{B}. \qquad
	\mathbf{B} = \mu_0 \bar{\bar{\mu}} \cdot \mathbf{H},
\end{equation}
где $\bar{\bar{\varepsilon}}$ и $\bar{\bar{\mu}}$ тензоры $3 \times 3$ диэлектрической и магнитной 
проницаемости соответсвенно. Рассмотрим преобразование координат из Декартового пространства 
$(x, y, z)$ в произвольное искривленное пространство, описываемое координатами $(q_1, q_2, q_3)$ с

\begin{equation}\label{e3}
	x = f_1(q_1, q_2, q_3) \qquad y = f_2(q_1, q_2, q_3) \qquad z = f_3(q_1, q_2, q_3).
\end{equation}
Матрица Якоби $\Lambda$ преобразования записывается как 

\begin{equation}\label{e4}
	\Lambda = 
	\begin{bmatrix}
	\frac{\partial x}{\partial q_1} & \frac{\partial x}{\partial q_2} & \frac{\partial x}{\partial q_3}\\
	\frac{\partial y}{\partial q_1} & \frac{\partial y}{\partial q_2} & \frac{\partial y}{\partial q_3}\\
	\frac{\partial z}{\partial q_1} & \frac{\partial z}{\partial q_2} & \frac{\partial z}{\partial q_3}\\
	\end{bmatrix}
\end{equation}
Длина линии элементра в преобразованном пространстве задается как 
$dl^2 = [dq_1, dq_2, dq_3]\mathbf{g}[dq_1, dq_2, dq_3]^T$, где $g = \Lambda^T\Lambda$ метрический тензор
пространства. Объем элемента пространства выражается как $dv = \det(\Lambda)dq_1dq_2dq_3$.

Тогда уравнения Максвелла у искривленном пространстве принимают их инвариантный вид (Ward and Pendry, 
1996)

\begin{equation}\label{e5}
	\nabla_q \times \mathbf{\hat{E}} = -\frac{\partial \mathbf{\hat{B}}}{\partial t}, \;\;
\nabla_q \times \mathbf{\hat{H}} = \frac{\partial \mathbf{\hat{D}}}{\partial t} + \mathbf{\hat{j}}, \;\;
\nabla_q \cdot \mathbf{\hat{D}} = \hat{\rho},\;\;
\nabla_q \cdot \mathbf{\hat{B}} = 0
\end{equation}
с новыми дополняющими уравнениями

\begin{equation}\label{e6}
	\mathbf{\hat{D}} = \epsilon_0 \hat{\bar{\bar{\varepsilon}}} \cdot \mathbf{\hat{E}}, \;\;
	\mathbf{\hat{B}} = \mu_0 \hat{\bar{\bar{\mu}}} \cdot \mathbf{\hat{H}},
\end{equation}
где все переменный в новой системе координат были обозначены с крышкой. Для того, чтобы сохранить
такую инвариантность уравнений Максвелла новые тензоры деэлектрической и магнитной проницаемости 
должны удовлетворять

\begin{equation}\label{e7}
	\hat{\bar{\bar{\varepsilon}}} = \det(\Lambda)(\Lambda)^{-1}\bar{\bar{\varepsilon}} \Lambda^{-T}, \;\;
	\hat{\bar{\bar{\mu}}} = \det(\Lambda)(\Lambda)^{-1}\bar{\bar{\mu}} \Lambda^{-T},
\end{equation}
здесь -Т обозначает транспонирование и обращение. Поля и источники в новой системе координат могут быть
непосредственно выведены из соответствующих распределений в исходной системе координат как

\begin{equation}\label{e8}
	\mathbf{\hat{E}} = \Lambda^T\mathbf{E}, \;\; 
	\mathbf{\hat{H}} = \Lambda^T\mathbf{H},
\end{equation}

\begin{equation}\label{e9}
\mathbf{\hat{j}} = \det(\Lambda)(\Lambda)^{-1}\mathbf{j}, \;\;
\hat{\rho} = \det(\Lambda)\rho.	
\end{equation}
Как видно из приведенных выше уравнений замена системы координат не спасает нас от решения в точности 
тех же уравнений, при условии, что диэлектрическая и магнитная проницаемости определены по разному.

Когда диэлектрическая и магнитная проницаемость среды в декартовой системе координат изотпропны,
недавно полученные диэлектрические и магнитные проницаемости, полученные в ур. \ref{e7} могут
быть переписаны как

\begin{equation}
	\hat{\bar{\bar{\varepsilon}}} = \det(\Lambda)g^{-1} \varepsilon,\;\;
	\hat{\bar{\bar{\mu}}} = \det(\Lambda)g^{-1} \mu.	
\end{equation}

Следует отметить, что преобразование координат может вполняться последовательно
\end{document}
