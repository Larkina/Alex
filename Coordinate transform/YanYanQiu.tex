\documentclass[12pt,a4paper]{article}
%\documentclass[17pt,russian]{extarticle}
\usepackage[utf8]{inputenc}
\usepackage[russian]{babel}
\usepackage[OT1]{fontenc}
\usepackage{amsmath}
\usepackage{amsfonts}
\usepackage{amssymb}
\usepackage{graphicx}
\usepackage{mathtools}
\usepackage[left=2cm, right=2cm, top=2cm, bottom=2cm]{geometry}

 \newcommand{\tit}[1]{\begin{center}{\bf{\Large #1}}\end{center}}
 \newcommand{\aut}[1]{\centerline{{\bf #1}}}
 \newcommand{\cityorg}[1]{\centerline{\it #1}}
 \newcommand{\email}[1]{\centerline{{\small e-mail: #1}}\vspace{\baselineskip}}
 \renewcommand{\normalsize}{\fontsize{16pt}{16}\selectfont}
 \renewcommand{\baselinestretch}{1.5}
\begin{document}

\sloppy
 
 \tit{Маскировка путем преобразования координат}
 \tit{Invisibility cloaking by coordinate transformation}
 \aut{Min Yan, Wei Yan and Min Qiu}
 \cityorg{Department of Microelectronics and Applied Physics,}
 \cityorg{Royal Institute of Technology (KTH), Electrum 229, 164 40 Kista, Sweden}
 \email{min@kth.se (M. Qiu)}

\section{Введение}
Появившиеся в последнее время искусственные электромагнитные (EM) материалы, называемыме 
\textit{метаматериалами} (Smith, Pendry and Wiltshire, 2004; Shalaev, 2006), открыли для нас
много способов взаимодействовать или управлять EM волнами. Были эксперементально продемонстрированы EM
явления, не существующие в природе. Отрицательный индекс преломления(Shelby, Smith and Schultz, 2001)
и отрицательный показатель пластины суперлинзы (Pendry, 2000) являются типичными примерами такого класса
спроектированных материалов. С помощью метаматериалов мы сповобны не только не только приспосабливать 
диалектрическую проницаемость и ее значения по желанию, на также точный контроль анизотропии 
метаматериалами и как зависят параметры от пространственного расположения. Хотя построение каждого 
отдельного блока метаматериала не является проблемой, помещение блоков в подходящем порядке, для 
достижения желаемого макроскопического оптического феномена остается загадкой. Говоря простым языком,
как с помощью деревьев различных цветов и размеров, имеющихся в нашем распоряжении, построить великолепный
лес? Недавно предложенный способ преобразования координат представляет собой идеальный рецепт для
решения такой задачи проектирования.

Данная статья посвящено разработке осбого типа EM устройства, мантии невидимки, полученной с помощью
техники преобразования координат. Структура нашей работы следущая: сначала, в разделе 2 обощается
теория преобразования координат для уравнения Максвелла. Принципы и построение скрывающих оболочек,
основанные на технике преобразования координат, представлены в разделе 3. В разделе 4 будет доказана
идеальная маскировка оболочек произвольной формы, используя доказаналитический анализ двухпериодной волны.
Будут внимательно рассмотрены физические параметры оболочек произвольной формы. После освещения
оболочек произвольной формы, мы будем детально изучать цилиндрические и сферические оболочки 
(секции 5 и 6). Специальное внимание будет уделено цилиндрической маскирующей оболочке, так как эта 
структура, возможно, является самой простой с точки зрения реализации. Некоторые практические вопросы,
касающиеся маскирующих оболочек, а также другие связанные исследования будут обсуждаться в разделе 7.
Наконец, в разделе 8 будут подведены итоги.

\section{Преобразование координат в электромагнетизме}
Теория трансформационной оптики уходит корнями в ковариционные свойства уравнений Максвелла. 
Лучшее математическое описание такого ковариционного свойства может быть дано с помощью диффиренциальной
геометрии (Post, 1962), аппарата, в общем используемого для разработки теории общей относительности
(Leonhardt and Philbin, 2006). Строгий вывод теории трансформационной оптики в четырехмерной пространстве
Минковского можно найти в Leonhardt and Philbin (2008). В этой работе мы напрямую используем наиболее
важные выводы в математически более доступной форме. Так как большинство приложений трансформационной 
оптики являются статическими или медленнодвижущимися по сравению с скоростью света, мы всегда можем
выбрать должным образом нашу рабочую область, и поэтому можно рассматривать только пространственное 
преобразование координат. В этой статье время не будет участвовать в преобразовании координат.

В плоском трехмерной евклидовом пространстве макроскопические уравнения Максвелла могут быть записаны как:

\begin{equation}\label{e1}
	\nabla \times \mathbf{E} = - \frac{\partial \mathbf{B}}{\partial t}, \qquad
	\nabla \times \mathbf{H} = \frac{\partial \mathbf{D}}{\partial t} + \mathbf{j}, \qquad
	\nabla \cdot \mathbf{D} = \rho, \;\;
	\nabla \cdot \mathbf{B} = 0.
\end{equation}
$\mathbf{E}$ и $\mathbf{H}$ --- электрические и магнитные поля, соответственно. $\mathbf{D}$ и 
$\mathbf{B}$ --- поток электрических и магнитных плотностей, соответственно. $\mathbf{j}$ плотность 
электрического тока, и $\rho$ плотность заряда. Уравнения Максвелла дополняются двумя соотношениями:

\begin{equation}\label{e2}
	\mathbf{D} = \epsilon_0 \bar{\bar{\varepsilon}} \cdot \mathbf{B}. \qquad
	\mathbf{B} = \mu_0 \bar{\bar{\mu}} \cdot \mathbf{H},
\end{equation}
где $\bar{\bar{\varepsilon}}$ и $\bar{\bar{\mu}}$ тензоры $3 \times 3$ диэлектрической и магнитной 
проницаемости соответсвенно. Рассмотрим преобразование координат из Декартового пространства 
$(x, y, z)$ в произвольное искривленное пространство, описываемое координатами $(q_1, q_2, q_3)$ с

\begin{equation}\label{e3}
	x = f_1(q_1, q_2, q_3) \qquad y = f_2(q_1, q_2, q_3) \qquad z = f_3(q_1, q_2, q_3).
\end{equation}
Матрица Якоби $\Lambda$ преобразования записывается как 

\begin{equation}\label{e4}
	\Lambda = 
	\begin{bmatrix}
	\frac{\partial x}{\partial q_1} & \frac{\partial x}{\partial q_2} & \frac{\partial x}{\partial q_3}\\
	\frac{\partial y}{\partial q_1} & \frac{\partial y}{\partial q_2} & \frac{\partial y}{\partial q_3}\\
	\frac{\partial z}{\partial q_1} & \frac{\partial z}{\partial q_2} & \frac{\partial z}{\partial q_3}\\
	\end{bmatrix}
\end{equation}
Длина линии элементра в преобразованном пространстве задается как 
$dl^2 = [dq_1, dq_2, dq_3]\mathbf{g}[dq_1, dq_2, dq_3]^T$, где $g = \Lambda^T\Lambda$ метрический тензор
пространства. Объем элемента пространства выражается как $dv = \det(\Lambda)dq_1dq_2dq_3$.

Тогда уравнения Максвелла у искривленном пространстве принимают их инвариантный вид (Ward and Pendry, 
1996)

\begin{equation}\label{e5}
	\nabla_q \times \mathbf{\hat{E}} = -\frac{\partial \mathbf{\hat{B}}}{\partial t}, \;\;
\nabla_q \times \mathbf{\hat{H}} = \frac{\partial \mathbf{\hat{D}}}{\partial t} + \mathbf{\hat{j}}, \;\;
\nabla_q \cdot \mathbf{\hat{D}} = \hat{\rho},\;\;
\nabla_q \cdot \mathbf{\hat{B}} = 0
\end{equation}
с новыми дополняющими уравнениями

\begin{equation}\label{e6}
	\mathbf{\hat{D}} = \epsilon_0 \hat{\bar{\bar{\varepsilon}}} \cdot \mathbf{\hat{E}}, \;\;
	\mathbf{\hat{B}} = \mu_0 \hat{\bar{\bar{\mu}}} \cdot \mathbf{\hat{H}},
\end{equation}
где все переменный в новой системе координат были обозначены с крышкой. Для того, чтобы сохранить
такую инвариантность уравнений Максвелла новые тензоры деэлектрической и магнитной проницаемости 
должны удовлетворять

\begin{equation}\label{e7}
	\hat{\bar{\bar{\varepsilon}}} = \det(\Lambda)(\Lambda)^{-1}\bar{\bar{\varepsilon}} \Lambda^{-T}, \;\;
	\hat{\bar{\bar{\mu}}} = \det(\Lambda)(\Lambda)^{-1}\bar{\bar{\mu}} \Lambda^{-T},
\end{equation}
здесь -Т обозначает транспонирование и обращение. Поля и источники в новой системе координат могут быть
непосредственно выведены из соответствующих распределений в исходной системе координат как

\begin{equation}\label{e8}
	\mathbf{\hat{E}} = \Lambda^T\mathbf{E}, \;\; 
	\mathbf{\hat{H}} = \Lambda^T\mathbf{H},
\end{equation}

\begin{equation}\label{e9}
\mathbf{\hat{j}} = \det(\Lambda)(\Lambda)^{-1}\mathbf{j}, \;\;
\hat{\rho} = \det(\Lambda)\rho.	
\end{equation}
Как видно из приведенных выше уравнений замена системы координат не спасает нас от решения в точности 
тех же уравнений, при условии, что диэлектрическая и магнитная проницаемости определены по разному.

Когда диэлектрическая и магнитная проницаемость среды в декартовой системе координат изотпропны,
недавно полученные диэлектрические и магнитные проницаемости, полученные в ур. \ref{e7} могут
быть переписаны как

\begin{equation}
	\hat{\bar{\bar{\varepsilon}}} = \det(\Lambda)g^{-1} \varepsilon,\;\;
	\hat{\bar{\bar{\mu}}} = \det(\Lambda)g^{-1} \mu.	
\end{equation}

Следует отметить, что преобразование координат может вполняться последовательно. Отношения между EM 
системой в начальной системе координат и в конечной системе могут быть охарактеризованы глобальной 
матрицей Якоби преобразования координат, которая вычисляется умножением матриц Якоби для каждого
индивидуального преобразования (Zolla, Guenneau, Nicolet and Pendry, 2007). Глобальна матрица Якоби
затем может быть применена для получения материалов и распределения полей в окончательной системе
координат. Одно полезное применение такой стратегии возникает, когда преобразование координат может
быть описано гораздо проще в некоторой системе координат, чем в декартовой. Например, в цилиндриечской
системе координат преобразование может включать только отображение радиальной компоненты, тоесть
из $(r, \theta, z)$ в $(r', \theta', z)$ с $\theta=\theta'$ и $z=z'$. Для интерпретации преобразования
в декартовой системе координат глобальная матрица Якоби 

\begin{equation}
	\Lambda = \Lambda_{xr}\Lambda_{rr'}\Lambda_{r'x'},
\end{equation}
где $\Lambda_{xr}$ характеризует переход из декартовой системы координат в цилиндрическую, $\Lambda_{rr'}$
из оригинальных цилиндрических координат в новый цилиндрические, и $\Lambda_{r'x'}$ обозначает переход
из новых цилиндрических координат обратно в декартову.

Есть два особенно важных приложения ковариационного свойства электродинамики, описанног выше. Во первых,
геометрия некоторых сложных структур может быть существенно упрощена, если они описываются в их 
``естесственных'' координатах. Такое альтернативное геометрическое описание сложных структур может 
способствовать более эффективному теоретическому рассмотрению EM проблемы, как правило, за счет более
сложных распределений физических параметров. Одним из примеров является численный случай с спирально
закрученным оптическим волноводом  (Nicolet, Zolla and Guenneau, 2004; Shyroki, 2008), в котором 
трехмерная задача упрощается до двухмерной. Второе применение ковариационного свойства заключается
в разработке оптических устройств, с помощью которых можно получить новые явления. Из уравнений,
представленных выше, новое EM поле, если интерпретировать со стороны декартовой системы координат,
будет казаться искаженной. Искажение поля характеризуется матрицей Якоби, и что в свою очередь 
характеризует как мы выбираем искривленную систему координат. Следует отметить, что структура, 
включающая физические параметры, полученные при помощи преобразования координат в искривленной ситеме
координат, преобразуется назад в декартову систему координат. Это так называемая 
``физическая интерпретация'' ковариационного свойства уравнений Максвелла (Schurig, Pendry and Smith, 
2006b). Когда вновь полученные физические праметры проинтерпретируюем в оригинальной прямоугольной 
системе координат, получим эффективную среду, которая называется \textit{трансформационной средной}.
Трансформационная среда имитирует эффект изогнутых пространственных координат и облегчает преломление 
света. Во многих случаях, искажение EM поля это то, чего пытаются достичь большинство оптических устройст
в. Типичный пример включает волноводный изгиб, где изменение направления луча желаемо, чтобы облегчить
плотную оптическую интеграцию и линзы, в которых пространственно расходящиеся поля от точечного 
источника снова фокусируются на области. Тот факт, что мы в состоянии отталкиваясь из определеннного 
желаемого ЕМ поведения волны получить некоторую сложную среду используя метод координатного 
преобразования говорит о том, что разработка многих новых устройств находится в пределах нашей 
досягаемости. Технология метаматериалов завершит последний шаг изготовления, в соответствии с получнными 
распределенияим физических параметров.

В опубликованной литературе существует несколько предложенных устройств, основанных на теории 
преобразования координат, а именно: маскирующая оболочка (Leonhardt, 2006; Pendry, Schurig and Smith, 
2006), идеальная линза (Leonhardt and Philbin, 2006; Schurig, Pendry and Smith, 2007; Tsang and Psaltis, 
2008; Kildishev and Shalaev, 2008; Kildishev and Narimanov, 2007; Yan, Yan and Qiu, 2008a), вращатель EM 
поля (Chen and Chan, 2007), EM концентраторы (Rahm, Schurig, Roberts, Cummer, Smith and Pendry, 2008b),
разделитель EM луа (Rahm, Cummer, Schurig, Pendry and Smith, 2008a), антенна (Kong, Wu, Kong, Huangfu, 
Xi and Chen, 2007) и ЕМ кротовые норы (Greenleaf, Kurylev, Lassas and Uhlmann, 2007a), и.т.д
Когда время так же учавствует в трансформации, как в пространстве Минковского, теория преобразования
координат также помогает объяснить EM аналог горизонта событий, так же, как и так называемый оптический
эффект Ааронова-Бома (Leonhardt and Philbin, 2006). В данной статье мы сфокусируемся в частности на ЕМ
масирующих устройствах. Не только потому, что цели ЕМ маскировки возродили ковариационные принципы
электродинамики в построении новых устройств, но также потому, что технология скрывающей оболочки сама
по себе представляет большой интерес во многих военных и гражданских приложениях. Маскировка остается
очень популярной идеей в мировой научной фантастике или фентезийных произведениях. Возможность устройства
EM маскировки недавно вдохновила теоретические исследования в звуковой маскировке (Milton, Briane and 
Willis, 2006; Cummer and Schurig, 2007; Cummer, Popa, Schurig, Smith, Pendry, Rahm and Starr, 2008; 
Norris, 2008; Chen, Yang, Luo and Ma, 2008), фазовых волнах (Zhang, Genov, Sun and Zhang, 2008d; 
Greenleaf, Kurylev, Lassas and Uhlmann, 2008) и даже поверхностных волн в жидкости (Farhat, Enoch, 
Guenneau and Movchan, 2008). 

Следует отметить, что некоторые исследования по маскировки существовали до работы Пендри и его соавторов
(Pendry, Schurig and Smith, 2006). Примечательно, что техника, похожая на преобразование координат была
применена для достижения невидиости в статическом пределе (Greenleaf, Lassas and Uhlmann, 2003). Более
того, были предложены техники маскировки, не основанные на подходе преобразования координат. Среди этих
работ плазмонный резонанс поверхности был использован для сокрытия маленьких частиц (смотри также Alu
and Engheta (2005); Milton and Nicorovici (2006); Alu and Engheta (2008)) и также в работе Miller (2006) 
сеть датчиков и сенсоров для обеспечения невидимости.

\section{Принципы и построение маскирующих оболочек}

\end{document}
