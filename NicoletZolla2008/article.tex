\documentclass[a4paper, 12pt]{article}
\usepackage{cmap}
\usepackage[utf8]{inputenc}
\usepackage[english, russian]{babel}
\usepackage[left=2cm, right=2cm, top=2cm, bottom=2cm]{geometry}
\usepackage{amsfonts,amssymb}
\usepackage{amsmath}
\usepackage{amsthm}
\usepackage{titlesec}
\usepackage{graphicx}
\usepackage{mathtools}
\usepackage{hyperref}

 \newcommand{\tit}[1]{\begin{center}{\bf{\Large #1}}\end{center}}
 \newcommand{\aut}[1]{\centerline{{\bf #1}}}
 \newcommand{\cityorg}[1]{\centerline{\it #1}}
 \newcommand{\email}[1]{\centerline{{\small e-mail: #1}}\vspace{\baselineskip}}
\providecommand{\keywords}[1]{\textbf{\textit{Ключевые слова:}} #1}
\newcommand{\norm}[1]{\left\lVert#1\right\rVert}
\newcommand{\normb}[1]{\left\lVert\textbf{#1}\right\rVert}

\begin{document}

\sloppy
\tit{Электромагнитный анализ цилиндрических оболочек произвольного поперечного
сечения}
 \aut{Ander Nicolet, Frederic Zolla, Sebastien Guenneau}

\begin{abstract}
Мы продолжим конструкции радиальных симметричных маскирующих оболочек
при помощи трансформационной оптики, как предложено Пендри и др.
чтобы скрыть цилиндр произвольного поперечного сечения.
Справедливость нашего подхода, основанного на методе Фурье, подтверждается
как аналитическими, так и численными результатами для оболочки, представляющую
собой невыпуклое поперечное сечение произвольной толщины. В первом случае,
мы можем вычислить функцию Грина линейного источника в преобразованных
координатах. Во втором случае, мы реализуем модель конечных элементов 
полной волны для цилиндрической антенны, излучающей $p$-поляризованное
электрическое поле при наличии F-образного объекта с потерями, 
окруженного оболочкой.
\end{abstract}

Метаматериалы, известные своими применениями для субволновых визуализаций \cite{1}-
\cite{3}, открыли новый путь в электромагнитной маскировке либо их
неодронодными анизотропными эффективными материальными параметрами
(трансформационная оптика \cite{4,5}), либо материалами с низким \cite{6} или отрицательным индексом преломления \cite{7}. Интересно, что невидимость
сохраняется в случае интенсивного ближнего поля \cite{8}, когда нарушается
картинка лучевой оптики. Тем не менее, первая эксперементальная реализация,
главным образом достигнутая в микроволновом режиме \cite{9}, предпологает,
что маскировка будет ограничена очень узким диапазоном частот. В оптическом
спектре, она так же будет необходимо диссипативна и дисперсионна.

В настоящей работе мы обсуждаем построение цилиндрической оболочки с произвольным 
поперечным сечением, описываемым двумя функциями, $R_1(\theta)$ и $R_2(\theta)$,
задающими зависящее от угла расстояние от начала координат. Эти функции отвечают,
соответсвенно, внутренней и внешней границе оболочки. Мы только будем
предпологать, что эти две границы могут быть представлены диффиренцируемыми
функциями, описываемыми, например, конечным разложением в ряд Фурье.
Таким образом, наш подход может быть применен не только для эллиптических оболочек
\cite{11,12}, но и для оболочек с гладкими невыпуклыми границами (но не квадратной 
формы \cite{13}).

Чтобы продемонстировать нашу методологию, рассмотрим оболочку, не обладающую ни
вращательной, ни трансляционной симметрией в поперечной плоскости. Сначала
мы ищучим параметры оболочки на ее неотрожающей внешней границе (см. рис. 1).
Затем вычислим полноволновую картинку (при помощи конечных элементов)
для объекта с потерями с острыми клиньями, окруженного оболочкой, при наличии
близко расположенной антенны (см. рис. 2 и 3). Наконец, мы сравним эти
численные результаты с анатилической моделью, вычислив функцию Грина
$p$-поляризованного линейного источника в соответствующей преобразованной
метрике (см. рис. 4). Это подтверждает, что маскировка осуществляется не только
для далекого поля, а также для близкого, где лучевая модель поля распадается.

Геометрическое преобразование, которое отражает поле во всей области 
$\rho \le R_2(\theta)$ в кольцевую область $R_1(\theta) \le \rho' le R_2(\theta)$
может быть записано как

\begin{equation}
	\begin{cases}
\rho'(\rho, \theta) = R_1(\theta)+\rho \frac{R_2(\theta)-R_1(\theta)}{R_2(\theta)}\\
\theta'=\theta, \qquad 0 < \theta \le 2\pi\\
z'=z, \qquad z\in \mathbb{R}
	\end{cases},
\end{equation}
где $0 \le \rho R_2(\theta)$. Отметим, что преобразование отражает поле
для $\rho > R_2(\theta)$ на себя тождественным преобразованием.

Эта замена координат, характеризуется преобразованием дифиренциалов 
при помощи якобиана:

\begin{equation}
	\mathbf{J}(\rho',\theta') = 
	\frac{\partial (\rho(\rho',\theta), \theta, z)}{\partial(\rho',\theta',z')}.
\end{equation}

С электромагнитной точки зрения, это изменение величин координат 
отображением однородной изотропной среды с скалярными диэлектрической
и магнитной проницаемостью $\epsilon$ и $\mu$ на материальные, описываемые
анизотропной неодронодной матрицами диэлетрической и магитной проницаемости
задается \cite{8,14}

\begin{equation}
	\underset{=}{\epsilon'} = \epsilon \mathbf{T}^{-1}, \qquad
	\underset{=}{\mu'} = \mu \mathbf{T}^{-1},
\end{equation}
здесь $\mathbf{T}=\mathbf{J}^T\mathbf{J}/\det(\mathbf{J})$ ~--- представление 
метрического тензора в так называемых растянутых радиальных координатах.

После некоторых элементарных алгебраических преобразований находим, что

\begin{equation}
	\mathbf{T}^{-1} = 
	\begin{pmatrix}
\frac{c_{12}^2+f_\rho^2}{c_{11}f_\rho\rho'} & -\frac{c_{12}}{f_\rho} & 0 \\[10pt]
	-\frac{c_{12}}{f_\rho} &  \frac{c_{11}\rho'}{f_\rho} & 0 \\[10pt]
	0 & 0 & \frac{c_{11}f_\rho}{\rho'}\\[10pt]
	\end{pmatrix},
\end{equation}
где $c_{11}(\theta')=R_2(\theta')/[R_2(\theta')-R_1(\theta')]$ и

\begin{equation*}
c_{12}(\rho',\theta)=[\rho'-R_2(\theta')]R_2(\theta')\frac{dR_1(\theta')}{d\theta'}	
\end{equation*}
\begin{equation}
	-\frac{[\rho'-R_1(\theta)]R_1(\theta')}{[R_2(\theta)-R_1(\theta)]^2}
	\frac{dR_2(\theta')}{d\theta'},
\end{equation}
для $R_1(\theta') \le R_2(\theta')$ c 

\begin{equation}
	f_\rho(\rho',\theta') = [\rho'-R_1(\theta')]
	\frac{R_2(\theta')}{R_2(\theta')-R_1(\theta')}.	
\end{equation}
В другой области $\mathbf{T}^{-1}$ переходит в тождественную матрицу
$[c_{11}=1, c_{12}=0 \text{ и } f_\rho' \text{ для } \rho'>R_2(\theta')]$.

Сейчас мы бы хотели посмотреть на электромагнитное поле, излучаемое 
проводом антненны-источника с центром в точке $\mathbf{r}_s$, при
наличии конечно проводящего объекта в форме буквы \"F\", когда он окружен
оболочной произвольной формы. Благодаря цилиндрической геометрии, 
задача разделяется на две поляризации. В $p$ поляризации
мы находим, что 

\begin{equation}
\nabla (\underset{=}{\mu_T^{'-1}}\nabla E_z)+\mu_0\epsilon_0\omega^2\epsilon'_{zz}
E_z = -i\omega I_s\mu_0\delta_{\mathbf{r}_s},	
\end{equation}
где $\mu_0\epsilon_0=c^{-2}$, $c$ ~--- скорость света в вакууме,
$\underset{=}{\mu_T^{'-1}}$ выступает в качестве верхнего блока диагольной части
$\underset{=}\mu'^{-1}$ [см. ур.(3) и (4)], и 
$\epsilon'_{zz}=\epsilon(c_{11}f_\rho/\rho'$. Также, $E_z$ и $I_s$ обозначают
соответсвенно только ненулевые компоненты продольного электрического поля
$\mathbf{E}_l=E_z(\rho,\theta)\mathbf{e}_z$ и заданный элекрический ток
$\mathbf{J}_s=I_s\delta_{\mathbf{s}}\mathbf{e}_z$ на проводе антенны.

Слабая форма этого уравнения была реализована в бесплатном пакете конечных
элементов GetDP \cite{15}, где были исользованы круглые прекрасно подобранные
слои для моделирования бесконечной внешней среды, окружающей оболочку.
Сначала мы рассмотрели цилиндрическую оболочку эллиптического поперечного сечения,
параметризованную как $\rho(\theta)=ab/\sqrt{a^2\cos^2(\theta)+b^2\sin^2(\theta)}$.
Мы проверили, что она производит в точность такие же траектории волн и профиль 
поля, как в \cite{12}, где схожие результаты были получены при помощи 
искривления пространства круглой цилиндрической оболочки. Кроме того,
мы восстановили шаблон волны из \cite{11}, где рассматривалась
эксцентричная эллиптическая кольцевая оболочка.

Чтобы получить общие формы может быть использовано конечное разложение
в ряд Фурье:

\begin{equation}
	\rho(\theta)=a_0+\sum\limits_{k=1}^{n}[a_k\cos(k\theta)+b_k\sin(k\theta)].
\end{equation}
Для иллюстрации (смотри рис. 2), рассмотрим оболочку с внутренней и внешней 
границей выражаемой в виде
\begin{equation*}
	R_1(\theta)=1+0.1\sin(\theta)+0.15\cos(2\theta)+0.2\sin(3\theta)
	+0.1\cos(4\theta)
\end{equation*}
\begin{equation}
	R_2(\theta) = 2-0.1\cos(2\theta)-0.15\cos(3\theta)+0.3\sin(3\theta)+
	0.2\cos(4\theta)
\end{equation}

Интересно исследовать параметры оболочки на ее внутенней неотражающей границе
$R_2(\theta)$. Это может быть сделано на основе анализа элементов
обратного метрического тензора $\mathbf{T}$ в полярных координатах.
Для начала отметим в рис. 1, что все недиагональные члены 
$(T^{-1})_{r\theta}=(T^{-1})_{\theta r}$ как правило, отличны от нуля,
в отличие от случая круглой оболочки, в котором $\mathbf{T}^{-1}$ ~--- 
диагональный. Здесь $\mathbf{T}^{-1}$ является диагональным только
для трех значений угла $\theta$, отвечающего шести точкам внешней границы,
в которых преобразованная система координат локально ортогональна.
Это демонстрирует, что предыдущий критерий для неотражающего интерфейса
$R_2(\theta)$, а именно, $T^{-1}_{\theta\theta}=T^{-1}_{zz}=1/
T^{-1}_{rr}, T^{-1}_{r\theta}=T^{-1}_{\theta r}=0$ (для круглого случая \cite{4}) 
и $T^{-1}_{zz}=1/T^{-1}_{rr}$ (для эксцентричного эллиптического случая \cite{11}),
могут быть дальше ослаблены. В целом, мы видим, что $-1<(T^{-1})_{r\theta} < 1$,

\begin{thebibliography}{99}
\bibitem{4}J. B. Pendry, D. Schurig, and D. R. Smith, Science 312, 1780 (2006).

\end{thebibliography}

\end{document}
