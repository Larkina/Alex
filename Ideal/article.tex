\documentclass[a4paper, 12pt]{article}
\usepackage{cmap}
\usepackage[utf8]{inputenc}
\usepackage[english, russian]{babel}
\usepackage[left=2cm, right=2cm, top=2cm, bottom=2cm]{geometry}
\usepackage{amsfonts,amssymb}
\usepackage{amsmath}
\usepackage{amsthm}
\usepackage{titlesec}
\usepackage{graphicx}
\usepackage{mathtools}

 \newcommand{\tit}[1]{\begin{center}{\bf{\Large #1}}\end{center}}
 \newcommand{\aut}[1]{\centerline{{\bf #1}}}
 \newcommand{\cityorg}[1]{\centerline{\it #1}}
 \newcommand{\email}[1]{\centerline{{\small e-mail: #1}}\vspace{\baselineskip}}
\providecommand{\keywords}[1]{\textbf{\textit{Ключевые слова:}} #1}
\newcommand{\norm}[1]{\left\lVert#1\right\rVert}
\newcommand{\normb}[1]{\left\lVert\textbf{#1}\right\rVert}
\newtheorem{lemma}{Лемма}
\begin{document}

\sloppy

 \tit{Идеальная цилиндрическая оболочка: совершенная, но чувствительная к малым возмущениям}
 \tit{Ideal Cylindrical Cloak: Perfect but Sensitive to Tiny Perturbations}
 \aut{Zhichao Ruan, Min Yan, Curtis W. Neff, and Min Qiu}
 \cityorg{Laboratory of Optics, Photonics and Quantum Electronics,} 
 \cityorg{Department of Microelectronics and Applied Physics,} 
 \cityorg{Royal Institute of Technology (KTH), Electrum 229, 16440 Kista, Sweden}
 \cityorg{Joint Research Center of Photonics of the}
 \cityorg{Royal Institute of Technology (Sweden) and Zhejiang University,}
 \cityorg{Zhejiang University, Yu-Quan, 310027 Hangzhou, People’s Republic of China}
 \email{min@kth.se}

\begin{abstract}
\end{abstract}

В последних работах обсуждался захватывающий вопрос экзотичесих материалов, невидимых для электромагнитных волн(
EM) \cite{1}-\cite{11}. Основываясь на преобразовании координат в уравнениях Максвелла Пендри и другие впервые
предложили мантию невидивку, которая может защищать объект внутри нее от обнаружения \cite{1}: когда 
электромагнитные волны проходят сквозь мантию невидимку, она отражает их, направляя вокруг объекта, затем
возвращает в первоначальное направление, не возмущая внешнее поле. Также недавно сообщалось о численных методах,
примененных для решения задачи EM, включающаю плащ невидимку \cite{6,9}, и экспериментальных результатов 
маскировки с использованием метаматериалов с упрощенными параметрами \cite{7}. Тем не менее, идеальный плащ-
невидимка не была подтверждена как совершенная оболочка, в связи с экстримальными физическими параметрами (ноль 
или бесконечность) в идеальной оболочке при приближении к внутренней границе.

\begin{thebibliography}{99}
\bibitem{1}J. B. Pendry, D. Schurig, and D. R. Smith, Science 312, 1780 (2006).
\bibitem{2}U. Leonhardt, Science 312, 1777 (2006).
\bibitem{3}A. Alu` and N. Engheta, Phys. Rev. E 72, 016623 (2005).
\bibitem{4}D. A. B. Miller, Opt. Express 14, 12457 (2006).
\bibitem{5}U. Leonhardt, New J. Phys. 8, 118 (2006).
\bibitem{6}S. A. Cummer, B. I. Popa, D. Schurig, D. R. Smith, and J. B. Pendry, Phys. Rev. E 74, 036621 (2006).
\bibitem{7}D. Schurig, J. J. Mock, B. J. Justice, S. A. Cummer, J. B. Pendry, A. F. Starr, and D. R. Smith, 
Science 314, 977 (2006).
\bibitem{8}G. W. Milton, M. Briane, and J. R. Willis, New J. Phys. 8, 248 (2006).
\bibitem{9}F. Zolla, S. Guenneau, A. Nicolet, and J. B. Pendry, Opt. Lett. 32, 1069 (2007).
\bibitem{10}W. Cai, U. K. Chettiar, A. V. Kildishev, and V. M. Shalaev, Nat. Photon. 1, 224 (2007).
\bibitem{11}H. Chen and C. T. Chan, Appl. Phys. Lett. 90, 241105 (2007).
\bibitem{12}H. C. van de Hulst, Light Scattering by Small Particles(Dover, New York, 1981).
\bibitem{13}D. Felbacq, G. Tayeb, and D. Maystre, J. Opt. Soc. Am. A 11, 2526 (1994).
\end{thebibliography}

\end{document}
